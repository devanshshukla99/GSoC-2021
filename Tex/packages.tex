
\usepackage[utf8]{inputenc}
\usepackage[a4paper, top=0.6in, right=0.8in, left=0.8in,bottom= 0.6in]{geometry}
% \usepackage{wrapfig, blindtext}
% \usepackage{graphicx}

%% packages

% \usepackage{blindtext} % needed for creating dummy text passages
%\usepackage{ngerman} % needed for German default language
\usepackage{amsmath} % needed for command eqref
\usepackage{amssymb} % needed for math fonts
\usepackage[
	colorlinks=true
	,breaklinks
	%,ngerman
	]{hyperref} % needed for creating hyperlinks in the document, the option colorlinks=true gets rid of the awful boxes, breaklinks breaks lonkg links (list of figures), and ngerman sets everything for german as default hyperlinks language
% \usepackage[hyphenbreaks]{breakurl} % ben�tigt f�r das Brechen von URLs in Literaturreferenzen, hyphenbreaks auch bei links, die �ber eine Seite gehen (mit hyphenation).
\usepackage{xcolor}
\definecolor{c1}{rgb}{0,0,1} % blue
\definecolor{c2}{rgb}{0,0.3,0.9} % light blue
\definecolor{c3}{rgb}{0.3,0,0.9} % red blue
\hypersetup{
    linkcolor={c1}, % internal links
    citecolor={c2}, % citations
    urlcolor={c3} % external links/urls
}
%\usepackage{cite} % needed for cite
\usepackage[round,authoryear]{natbib} % needed for cite and abbrvnat bibliography style
\usepackage[nottoc]{tocbibind} % needed for displaying bibliography and other in the table of contents
\usepackage{graphicx} % needed for \includegraphics 
\usepackage{longtable} % needed for long tables over pages
\usepackage{bigstrut} % needed for the command \bigstrut
\usepackage{enumerate} % needed for some options in enumerate
\usepackage{todonotes} % needed for todos
\usepackage{imakeidx} % needed for creating an index
\usepackage{fancyhdr}
\usepackage{float}

\pagestyle{fancy}
\fancyhf{}

\usepackage{bm}
\usepackage{tikz}
\usepackage{booktabs}
\usepackage{sectsty}
\usepackage{titlesec}

% ------------ Header/Footer hr ------------ %
\renewcommand{\headrulewidth}{0pt}
\renewcommand{\footrulewidth}{0.5pt}
\renewcommand{\arraystretch}{1,2}
% ------------ Header/Footer hr ------------ %

% ------------- Indentation ---------------- %
\setlength{\parindent}{0pt}
\setlength{\parskip}{10pt}
\setlength{\columnsep}{.5in}
% ------------- Indentation ---------------- %

% ------------------- subsubparagraph ------------------ %
\titleclass{\subsubparagraph}{straight}[\subparagraph]
\newcounter{subsubparagraph}
\renewcommand{\thesubsubparagraph}{\Alph{subsubparagraph}}
\titleformat{\subsubparagraph}[runin]{\normalfont\normalsize\bfseries}{\thesubsubparagraph}{1em}{}
\titlespacing*{\subsubparagraph} {\parindent}{3.25ex plus 1ex minus .2ex}{1em}
% ------------------- subsubparagraph ------------------ %

% ------------------- Table ------------------ %
\usepackage{array}
\usepackage{ragged2e}
\usepackage{microtype}
\newcolumntype{P}[1]{>{\raggedright\arraybackslash}p{#1}}
\newcolumntype{C}[1]{>{\centering\arraybackslash}p{#1}}
% ------------------- Table ------------------ %

% ------------------ Math -------------------------%
\thickmuskip=12mu plus 5mu minus 1mu            % Space after = sign
\medmuskip=8mu plus 2mu minus 3mu               % Space bt the Binary Operators
\thinmuskip=8mu          % Space after the Integral and comma, uniary Operators, I guess
% ------------------ Math -------------------------%

% ------------------ Section and Subsection -------------------------%
\titlespacing*{\section}{0pt}{*0.8}{*0.8}
\titlespacing*{\subsection}{0pt}{*0.8}{*0.8}
\titlespacing*{\subsubsection}{0pt}{*0.3}{*0.3}

\renewcommand{\thesection}{\Roman{section}} 
\renewcommand{\thesubsection}{\thesection.\Roman{subsection}}

\titleformat{\section}[block]{\large\bfseries\filcenter\hspace*{-0.5cm}\Roman{section}.}{}{1em}{}
\titleformat{\subsection}[block]{\bfseries\filcenter\hspace*{-0.5cm}\Roman{section}.\Roman{subsection}}{}{1em}{}
% ------------------ Section and Subsection -------------------------%
